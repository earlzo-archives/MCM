\thispagestyle{empty}

\vspace{-0.7cm}
\begin{footnotesize}
    %\begin{center}
        % Summary 说明
    %    \bf{2014 Mathematical Contest in Modeling (MCM) Summary Sheet}\\
    %\end{center}

    \begin{center}
        \textbf{Summary}
    \end{center}
        % Summary 正文
\par
In our paper, we focus on addressing problems of water quality forecast and the building of the lake’s evaluation system. Then we apply them into Chao Hu. Lastly, we analyze the results, meanwhile, putting forward suggestions on the improvement of the land management.\par
First and foremost, this paper mainly places emphasis on the nitrogen and phosphorus input as the criterion of lake, then a model for predicting water quality can be divided into two sections. We formulate an export coefficient model accounting for factors influencing the output of N and P in order to estimate and forecast nitrogen and phosphorus load under the different land use. On the basis of obtaining N and P outputs, BP neural network model is built by combining with indictors as inputs like temperature, PH and so on to predict potentially-toxic algal blooms. To ground this model in reality, we incorporate 91 groups data collected from the websites for train and 19 groups data used for the simulation. The simulation results agreeing well with real situation indicate that the model is efficient and reliable.\par
Secondly, a strategy based on analytic hierarchy process is proposed to build lake evaluation model. We collect the values of seven indexes judging lakes through local knowledge and expertise. Then we use AHP to determine the weight of seven factors and finally evaluate Chao Hu successfully. We draw conclusion that Chao Hu lies inⅢ,middle level.\par
Eventually, the sensitivity analysis of the evaluation model is carried out to ensure the utility. It is found that the model is shortage of insufficient stability by analyzing the sensitivity of environmental awareness, for lakes evaluation is a complicated system and cannot be described by simple indexes. So the model need to be improved.\par
\end{footnotesize}


\newpage
\thispagestyle{fancy}
\newpage
\setcounter{page}{1}
\vspace{2cm}

%\begin{center}
%    \begin{normalsize}
%Abstract Title
%    \end{normalsize}\\
%\end{center}

\begin{abstract}
    \begin{footnotesize}{}
% Abstract 正文
In our paper, we focus on addressing problems of water quality forecast and the building of the lake’s evaluation system. Then we apply them into Chao Hu. Lastly, we analyze the results, meanwhile, putting forward suggestions on the improvement of the land management.\par
First and foremost, we build a model to predict the water quality and potentially-toxic algal blooms. We divide the model into two sections. The first is the Export Coefficient Model, which is used to estimate and forecast nitrogen and phosphorus load under the different land-use. On the basis of obtaining N and P outputs, BP neural network model is built by combining with indictors as inputs like temperature, PH and so on to predict potentially-toxic algal blooms. To ground this model in reality, we incorporate 91 groups data collected from the websites for train and 19 groups data used for the test. The data of Chao Hu are used for simulation. The simulation results agree well with real situation indicating that the model is efficient and reliable.\par
Secondly, we build the lake evaluation model based on analytic hierarchy process. We collect the values of seven indexes judging lakes through local knowledge and expertise. Then we use AHP to determine the weight of seven factors and finally evaluate Chao Hu successfully. We draw conclusion that Chao Hu lies in Ⅲ,belonging to middle level.\par
Eventually, the sensitivity analysis of the evaluation model is carried out to ensure the utility. It is found that the model is shortage of insufficient stability by analyzing the sensitivity of environmental awareness, for lakes evaluation is a complicated system and cannot be described by simple indexes. So the model need to be improved.\par
    \end{footnotesize}
\end{abstract}

